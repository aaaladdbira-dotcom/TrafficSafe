% Traffic Accident Information System - Technical Report
\documentclass[a4paper,11pt]{article}
\usepackage[utf8]{inputenc}
\usepackage[T1]{fontenc}
\usepackage{lmodern}
\usepackage{geometry}
\geometry{margin=1in}
\usepackage{setspace}
\onehalfspacing
\usepackage{graphicx}
\usepackage{hyperref}
\usepackage{amsmath}
\usepackage{booktabs}
\usepackage{caption}
\usepackage{listings}
\usepackage{microtype}
\usepackage{enumitem}
\usepackage{fancyhdr}
\usepackage{multicol}
\usepackage{color}

% Header
\pagestyle{fancy}
\fancyhf{}
\lhead{Traffic Accident Information System}
\rhead{\thepage}
\renewcommand{\headrulewidth}{0.4pt}

\title{Technical Report\\Web Services Project\\Traffic Accident Information System}
\author{Alaa Dbira}
\date{January 17, 2026}

\begin{document}

\maketitle

\begin{abstract}
Accurate, timely collection and analysis of traffic-accident data is essential for improving
road safety and enabling evidence-based interventions. This project implements a modular
web platform to report, manage, visualize, and analyze traffic accidents. The system offers
secure REST APIs, a server-rendered UI with geospatial visualization, import/export
workflows, audit logging, real-time notifications, and operational tooling for deployment
and maintenance. The application is hosted at \url{https://trafficsafe.onrender.com/}. This report documents motivation, architecture, API design, implementation,
testing, deployment, results, and future work.
\end{abstract}

\tableofcontents
\clearpage

% =====================
% 0.1 Abstract (already present)
% =====================

% =====================
% 1 Introduction
% =====================
\section{Introduction}
\subsection{Motivation}
Road traffic accidents are a leading cause of mortality and economic loss. Centralized,
structured incident data improves response, supports analysis, and guides policy. The
Traffic Accident Information System aims to provide an accessible, auditable platform to
collect incident details, attach reports and media, and provide analytics and maps for
stakeholders.

\subsection{Problem Statement}
Current reporting is fragmented and often lacks geolocation, standardized fields, or
attachments. This reduces the usefulness of datasets for trend analysis, hotspot detection,
and policymaking.

\subsection{Objectives}
\begin{itemize}
  \item Provide RESTful APIs for accident CRUD operations and reporting.
  \item Implement role-based workflows for reporters, operators, and government reviewers.
  \item Deliver a dashboard with KPIs, time-series charts, and a Leaflet map.
  \item Support CSV import/export and file attachments.
  \item Ensure auditability, security, and deployability.
\end{itemize}

\subsection{Scope}
The system includes backend APIs, server-rendered UI, basic analytics, import/export,
WebSocket notifications, and DB migration support. It is designed to run on SQLite for
development and PostgreSQL in production.

\subsection{Methodology}
The project follows an agile methodology with iterative development, regular testing, and
feedback loops. Key phases included requirement analysis, system design, implementation,
testing, and deployment.

\subsection{Conclusion}
The Traffic Accident Information System addresses critical gaps in road safety data
collection and analysis. By leveraging modern web technologies and following best
practices in software development, the system is poised to deliver significant
value to stakeholders. The application is publicly hosted at \url{https://trafficsafe.onrender.com/}, making it accessible for demonstration and real-world use.

% =====================
% 2 System Architecture
% =====================
\section{System Architecture}
\subsection{Overview}
The application follows a layered architecture: presentation (UI), API layer, persistence,
real-time layer, and utilities for imports and audits. Components are organized under
`models/`, `resources/`, `ui/`, `schemas/`, and `utils/`.

\subsection{Backend Architecture}
The backend uses an application factory pattern (\texttt{create\_app()} in \texttt{app.py})
to initialize extensions and register blueprints. Extensions include `db`, `jwt`, `migrate`,
`limiter`, and `socketio`. Request timing and audit logs are captured for API calls.

\subsection{Database Design}
Relational schema with core tables: \textbf{users}, \textbf{accidents}, \textbf{accident\_reports},
\textbf{import\_batches}, \textbf{uploaded\_files}, and \textbf{audit\_logs}.

\subsubsection{Database Tables}
\begin{itemize}
  \item \textbf{users}: Stores user information and roles.
  \item \textbf{accidents}: Contains accident details and metadata.
  \item \textbf{accident\_reports}: Holds narrative reports and attachments for accidents.
  \item \textbf{import\_batches}: Tracks CSV import batches and their status.
  \item \textbf{uploaded\_files}: Manages file attachments for accidents and reports.
  \item \textbf{audit\_logs}: Records API access and data changes for auditing.
\end{itemize}

\subsubsection{SQL Schema Excerpt}
\begin{verbatim}
CREATE TABLE accidents (
  id INTEGER PRIMARY KEY AUTOINCREMENT,
  occurred_at DATETIME NOT NULL,
  latitude REAL,
  longitude REAL,
  severity VARCHAR(32),
  cause VARCHAR(128),
  governorate VARCHAR(64),
  delegation VARCHAR(64),
  batch_id INTEGER,
  created_at DATETIME DEFAULT CURRENT_TIMESTAMP
);
\end{verbatim}

\subsubsection{Relationships Between Tables}
\begin{itemize}
  \item \textbf{users} to \textbf{accidents}: One-to-many (a user can report multiple accidents).
  \item \textbf{accidents} to \textbf{accident\_reports}: One-to-many (an accident can have multiple reports).
  \item \textbf{import\_batches} to \textbf{accidents}: One-to-many (a batch can contain multiple accidents).
\end{itemize}

\subsection{Data Collection}
Data sources: manual submissions, CSV imports, and optional integrations with external feeds.
Imports validate and normalize coordinates, timestamps, and categories; invalid rows are
reported per-batch.

\subsubsection{Primary Data Sources}
\begin{itemize}
  \item Manual reports by users via the web interface.
  \item CSV files uploaded by authorized personnel.
  \item Real-time data from external APIs (e.g., weather, traffic).
\end{itemize}

\subsubsection{Data Integration and Processing}
\begin{itemize}
  \item CSV imports are processed by the import pipeline, which validates, transforms, and loads data.
  \item External API data is fetched and integrated using scheduled tasks and webhooks.
\end{itemize}

\subsection{Authentication and Security}
JWT tokens issued at login are used for API access; OAuth providers can be enabled. Security
measures include password hashing, RBAC, token expiration, rate limiting, input validation,
upload size checks, and audit logging.

\subsubsection{Authentication Mechanism}
\begin{itemize}
  \item Users authenticate via email/password or social login (OAuth).
  \item JWT tokens are issued upon successful login and are required for subsequent API requests.
\end{itemize}

\subsubsection{Security Measures}
\begin{itemize}
  \item Passwords are hashed using a strong, adaptive algorithm (e.g., bcrypt).
  \item Role-based access control (RBAC) restricts actions based on user roles (e.g., reporter, reviewer).
  \item API rate limiting protects against abuse and denial-of-service attacks.
  \item Input validation and sanitization prevent injection attacks and ensure data integrity.
  \item File uploads are scanned for malware, and size limits are enforced.
  \item Audit logs record all access and changes to sensitive data, enabling accountability and traceability.
\end{itemize}

\subsection{External API Integrations}
\subsubsection{Weather API Integration}
(if any) Integration with external weather APIs to enrich accident data with contextual information
(e.g., weather conditions at the time of the accident).

\subsubsection{Geospatial/Map API Integration}
Integration with mapping services (e.g., Leaflet, Google Maps) for geocoding, reverse geocoding,
and displaying accident locations on maps.

\subsection{Deployment and Scalability}
Containerized deployment via Docker; use PostgreSQL, Redis, and Gunicorn/NGINX for
production scalability.

\subsubsection{Containerized Deployment with Docker}
\begin{itemize}
  \item The application is packaged as a Docker container, including all dependencies and configuration.
  \item Docker Compose is used to define and manage multi-container applications (e.g., web server, database, cache).
\end{itemize}

\subsubsection{CI/CD}
Continuous integration and deployment pipelines are configured to automate testing, building,
and deployment of the application to various environments (e.g., development, staging, production).

\section{API Endpoints}
\subsection{Authentication and User Management}
\texttt{POST /api/auth/login} -- returns JWT.\\
\texttt{POST /api/auth/register} -- create user.

\subsection{Accident Management}
\texttt{GET /api/accidents} -- list with filters (date range, governorate, severity, bbox).
\texttt{POST /api/accidents} -- create accident.
\texttt{GET /api/accidents/{id}} -- detail; \texttt{PUT/PATCH}, \texttt{DELETE} supported.

\subsection{Report Management}
\texttt{POST /api/accidents/{id}/reports} -- add narrative report.

\subsection{File Uploads}
\texttt{POST /api/upload} -- attach files.

\subsection{Import/Export}
\texttt{POST /api/import} -- CSV upload (creates ImportBatch).
\texttt{GET /api/export} -- export filtered CSV.

\subsection{Stats and Dashboard}
\texttt{GET /api/stats/kpis}, \texttt{GET /api/stats/accidents/by\_month}, etc.

\subsection{Error Handling and Response Codes}
Consistent error response format with appropriate HTTP status codes. Common errors include
400 (Bad Request), 401 (Unauthorized), 403 (Forbidden), 404 (Not Found), and 500 (Internal Server Error).

% =====================
% 3 Implementation
% =====================
\section{Implementation}
\subsection{Technology Stack}
Flask, SQLAlchemy, Alembic, Flask-JWT-Extended, Flask-SocketIO, Flask-Smorest, Jinja2,
Chart.js, Leaflet, Docker, Gunicorn, NGINX.

\subsection{Project Structure}
\begin{verbatim}
traffic-accident/
  app.py
  models/
  resources/
  ui/
  schemas/
  static/
  templates/
  migrations/
  utils/
\end{verbatim}

\subsection{Application Setup}
\texttt{app.py} exposes \texttt{create\_app()} and a module-level \texttt{application} for WSGI.
At startup, the app creates missing auxiliary tables and adds backward-compatible columns
for SQLite deployments.

\subsection{Database Configuration}
Database connection settings are configured via environment variables (e.g., `DATABASE_URL`).
SQLAlchemy is used for ORM and database migrations are managed with Alembic.

\subsection{Authentication System}
The authentication system is built using Flask-JWT-Extended, which provides JWT creation,
verification, and refreshing. The user registration and login endpoints are secured and
validate input data.

\subsection{Import Pipeline}
CSV import creates \texttt{ImportBatch}, validates rows, inserts valid records, and returns a
row-level report. For large batches, move processing to background workers (Celery/RQ) and
report progress via SocketIO.

\subsection{Real-time Notifications}
Real-time notifications are implemented using Flask-SocketIO, allowing the server to push
updates to connected clients (e.g., progress of import jobs, new accident reports).

\subsection{UI and Visualizations}
Dashboard presents KPI cards, trend charts (Chart.js), and a Leaflet map with clustering and
popups. UI blueprints proxy API calls and attach session credentials to AJAX to avoid
exposing tokens.

\subsection{Observability and Logging}
Application performance and errors are monitored using logging and observability tools.
Structured logs are generated for API requests, errors, and important events (e.g., imports,
exports).

\subsection{Example Code Snippets}
\begin{lstlisting}[language=Python]
class Accident(db.Model):
    id = db.Column(db.Integer, primary_key=True)
    occurred_at = db.Column(db.DateTime, nullable=False)
    latitude = db.Column(db.Float)
    longitude = db.Column(db.Float)
    severity = db.Column(db.String(32))
    cause = db.Column(db.String(128))
    governorate = db.Column(db.String(64))
    delegation = db.Column(db.String(64))
    batch_id = db.Column(db.Integer, nullable=True)
    created_at = db.Column(db.DateTime, default=db.func.now())
\end{lstlisting}

\subsection{Conclusion}
The implementation of the Traffic Accident Information System leverages modern technologies
and best practices to deliver a robust, scalable, and secure solution for managing traffic
accident data. The use of containerization, RESTful APIs, and real-time communication
ensures that the system is well-equipped to handle the needs of its users and can
easily adapt to future requirements.

% =====================
% 4 Testing and Validation
% =====================
\section{Testing and Validation}
\subsection{Testing Approach}
Unit tests for model logic and validations; integration tests with Flask test client for auth
flows and endpoints; manual UI checks for map rendering.

\subsection{Example Test}
\begin{lstlisting}[language=Python]
def test_accident_creation(client):
    response = client.post('/api/accidents', json={
        'occurred_at': '2023-01-01T12:00:00',
        'latitude': 34.05,
        'longitude': -118.25,
        'severity': 'High',
        'cause': 'Collision',
        'governorate': 'California',
        'delegation': 'Los Angeles'
    })
    assert response.status_code == 201
    assert response.json['id'] is not None
\end{lstlisting}

\subsection{Data Validation}
Data validation is performed at multiple levels: client-side (UI), API (request/response),
and database (constraints, types). Invalid data is rejected with appropriate error messages.

\subsection{Results and Evaluation}
Testing has shown that the system meets its functional requirements and performs well
under expected load conditions. Some minor issues were identified and addressed during
testing.

\subsection{Limitations}
Current limitations include the need for manual intervention in some error cases, limited
geospatial analysis capabilities, and reliance on external APIs for certain data.

\subsection{Conclusion}
The testing and validation phase has confirmed the stability and reliability of the Traffic
Accident Information System. Ongoing monitoring and periodic testing will be essential
to maintain system integrity and performance.

% =====================
% 5 Deployment
% =====================
\section{Deployment}
\subsection{Local Development}
For local development, Docker Compose is used to spin up the application and its dependencies
(database, cache, etc.) in isolated containers. Environment variables are configured in a
`.env` file.

\subsection{Production Recommendations}
In production, it is recommended to use separate containers for the web server, database,
and cache. Use a reverse proxy (e.g., NGINX) to handle SSL termination and route requests
to the appropriate service.

\subsection{CI/CD}
Continuous integration and deployment pipelines are configured to automate the process of
building, testing, and deploying the application. Code changes are automatically tested and
deployed to staging, with manual approval required for production deployments.

\subsection{Backup and Monitoring}
Regular backups of the database and uploaded files are essential. Monitoring should be
configured to alert on critical issues (e.g., high error rates, slow performance).

\subsection{Deployment Checklist}
\begin{enumerate}
  \item Configure secrets and `DATABASE_URL` in environment variables.
  \item Run Alembic migrations (Postgres recommended).
  \item Serve with Gunicorn behind NGINX and enable HTTPS.
  \item Configure Redis for SocketIO and caching.
  \item Schedule backups for DB and uploaded files.
\end{enumerate}

% =====================
% 6 Evaluation, Risks and Future Work
% =====================
\section{Evaluation, Risks and Future Work}
\subsection{Evaluation}
The Traffic Accident Information System was evaluated based on its functionality, performance,
security, and usability. Feedback from initial users has been positive, highlighting the
system's ease of use and the value of its analytics features.

\subsection{Risk Analysis}
Key risks include data quality, privacy concerns, and system scalability. Mitigation strategies
include robust data validation, anonymization of personal data, and designing for horizontal
scalability.

\subsection{Future Enhancements}
Future work includes predictive analytics, mobile clients, and integration with emergency services.
Additional data sources and advanced analytics capabilities will be explored to enhance
the system's value.

\subsection{Conclusion}
The Traffic Accident Information System represents a significant advancement in the
collection and analysis of traffic accident data. By addressing the identified risks and
continuously evolving the system, we can further enhance road safety and response
efforts.

% =====================
% Bibliography
% =====================
\section*{References}
\begin{thebibliography}{9}
\bibitem{flask} Flask documentation. \url{https://flask.palletsprojects.com/}
\bibitem{sqlalchemy} SQLAlchemy. \url{https://www.sqlalchemy.org/}
\bibitem{leaflet} Leaflet. \url{https://leafletjs.com/}
\bibitem{chartjs} Chart.js. \url{https://www.chartjs.org/}
\end{thebibliography}

% =====================
% Appendices
% =====================
\appendix
\section{Appendix A: Hosting}
The Traffic Accident Information System is deployed and publicly accessible at:

\begin{itemize}
  \item \textbf{URL:} \url{https://trafficsafe.onrender.com/}
  \item The application is hosted on Render, a cloud platform that provides scalable and reliable web hosting for modern web applications.
  \item The deployment process uses Docker containers and environment variables for configuration, ensuring portability and ease of updates.
  \item Users can access the full functionality of the system, including reporting, analytics, and dashboard features, via this URL.
\end{itemize}

\end{document}
